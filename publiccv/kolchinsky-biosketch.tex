\documentclass[margin,line,centered,12pt]{res}
\usepackage[utf8]{inputenc}
\usepackage[unicode=true]{hyperref}
\setlength{\parskip}{.35em}
\newsectionwidth{8em}
\usepackage{mathptmx}
\usepackage{enumitem}
\usepackage{helvet}
\usepackage{courier}

\pagestyle{empty} 


\makeatletter
%%%%%%%%%%%%%%%%%%%%%%%%%%%%%% User specified LaTeX commands.
\usepackage{xcolor}
\definecolor{darkblue}{rgb}{0,0,0.5}
\usepackage{hyperref}

\hypersetup{colorlinks=true,linkcolor=darkblue,urlcolor=darkblue}

% \oddsidemargin -.35in
% \evensidemargin -.35in

%\textwidth=5.5in

\itemsep=0in
\parsep=0in

% if using pdflatex:
\setlength{\pdfpagewidth}{\paperwidth}
\setlength{\pdfpageheight}{\paperheight} 

\newenvironment{list1}{
  \begin{list}{\ding{113}}{%
      \setlength{\itemsep}{0in}
      \setlength{\parsep}{0in} \setlength{\parskip}{0in}
      \setlength{\topsep}{0in} \setlength{\partopsep}{0in} 
      \setlength{\leftmargin}{0.17in}}}{\end{list}}
\newenvironment{list2}{
  \begin{list}{$\bullet$}{%
      \setlength{\itemsep}{0in}
      \setlength{\parsep}{0in} \setlength{\parskip}{0in}
      \setlength{\topsep}{0in} \setlength{\partopsep}{0in} 
      \setlength{\leftmargin}{0.2in}}}{\end{list}}

\sloppy

\makeatother

\begin{document}
\name{Biographical sketch of Artemy Kolchinsky \vspace*{.1in}}
\begin{resume}

% \section{\textsc{Contact}}

% \begin{ncolumn}{2}
% Santa Fe Institute & E-mail: \href{mailto:artemyk@gmail.com}{artemyk@gmail.com}\\
% 1399 Hyde Park Rd. & \href{https://www.sites.google.com/site/artemyk/}{www.sites.google.com/site/artemyk/}\\
% Santa Fe, NM 87501 & \href{https://scholar.google.com/citations?user=RmRwJJIAAAAJ\&sortby=pubdate}{Google Scholar} / 
% GitHub: \href{https://github.com/artemyk/}{@artemyk}
% \end{ncolumn}

\section{\textsc{Education}}

\textbf{Indiana University}, Bloomington, IN, 2015\\
Ph.D. in Informatics (focus in Complex Systems), Minor in Cognitive
Science\\
Thesis: ``Measuring Scales: Integration and Modularity in Complex
Systems''

\textbf{New York University}, New York, NY, 2004 \\
B.A. \emph{magna cum laude}, Individualized Study (concentration in Complex
Systems)


%\vspace{10pt}

\section{\textsc{Academic Positions}}

\textbf{Santa Fe Institute}, Santa Fe, NM, Dec 2015-Present, Postdoctoral fellow

%\textbf{Massachusetts Institute of Technology}, Cambridge, MA, 2015-2016, Visiting scientist

\textbf{Instituto Gulbenkian de Ciência}, Portugal, 2009-2010, Visiting researcher

%\item \textbf{Research assistant}, NIH-funded project on text mining biomedical literature in PubMed. Indiana University, Bloomington, IN. Supervisor: Luis M. Rocha. 2014-2015
%\item \textbf{Research assistant}, Project for exploring collaborative filtering methods for large-scale job recommendation. Indiana University, Bloomington, IN. Supervisor: Luis M. Rocha. 2012-2013
%\item \textbf{Research assistant}, Indiana University-funded project on discovering drug-drug interactions by text mining biomedical literature. Indiana University, Bloomington, IN. Supervisor: Luis M. Rocha. 2011-2012
%\item \textbf{Freelance programmer}, Developed a variety of database-backed
%web applications. 1998-2007.

% \section{\textsc{Industry}}

% \textbf{LinkedIn Corporation}, Mountain View, CA, Summer 2014\\
% Data science internship. Supervisor: Mathieu Bastian

%\vspace{10pt}



\section{\textsc{Top 5 most relevant publications}}

D.H. Wolpert and \textbf{A. Kolchinsky}, ``The thermodynamics of computing with circuits'', \emph{New Journal of Physics}, 2020. \href{https://iopscience.iop.org/article/10.1088/1367-2630/ab82b8}{link} \href{https://arxiv.org/abs/1806.04103}{arxiv}


\textbf{A. Kolchinsky} and B. Corominas-Murtra, 
``Decomposing information into copying versus transformation'', 
\emph{Royal Society Interface}, 2020. 
\href{https://royalsocietypublishing.org/doi/10.1098/rsif.2019.0623}{link}
\href{https://arxiv.org/abs/1903.10693}{arxiv}




\textbf{A. Kolchinsky} and D.H. Wolpert, 
``Semantic information, autonomous agency, and nonequilibrium statistical physics'', 
\emph{Royal Society Interface Focus}, 2018.
\href{http://rsfs.royalsocietypublishing.org/content/8/6/20180041}{link} 
\href{https://arxiv.org/abs/1806.08053}{arxiv} 
\href{https://github.com/artemyk/semantic_information/}{code}


\textbf{A. Kolchinsky}, D.H. Wolpert, ``Dependence of dissipation on the initial distribution over states'',
\emph{Journal of Statistical Mechanics}, 083202, 2017. 
\href{http://dx.doi.org/10.1088/1742-5468/aa7ee1}{link}
\href{https://arxiv.org/abs/1607.00956}{arxiv}


\textbf{A. Kolchinsky}, A.J. Gates, L.M. Rocha, ``Modularity and
the spread of perturbations in complex dynamical systems,'' \emph{Physical Review E}, 2015. \href{https://journals.aps.org/pre/abstract/10.1103/PhysRevE.92.060801}{link} 
\href{http://arxiv.org/abs/1509.04386}{arxiv} 
%\href{https://github.com/artemyk/perturbationmodularity/}{code}






% Parada, F.J., A. Kolchinsky, Rossi, Puce, A. ``EEG phase-coupling
% dynamics in apparent motion perception,'' \emph{Organization for
% Human Brain Mapping 2013} (Seattle, WA), 2013.

%Parada, F.J., A. Kolchinsky, Rossi, A., Busey, T., Sporns,
% O., Puce, A. ``From visual awareness to perceptual decision-making:
% neural correlates of top-down and bottom-up dynamics,'' \emph{MBI
% Rhythms and Oscillations Workshop} (Columbus, OH), 2013.

%Rossi, A., Parada, F.J., A. Kolchinsky, Puce, A. ``Neural
%responses to changes in social attention depicted by biological motion
%stimuli,'' \emph{Organization for Human Brain Mapping 2012} (Beijing,
%China), 2012.

\section{\textsc{Other relevant publications and preprints}}

\textbf{A. Kolchinsky}, D.H. Wolpert, ``Thermodynamic costs of Turing Machines'', arXiv:1912.04685, 2019. \href{https://arxiv.org/abs/1912.04685}{arxiv}

D.H. Wolpert, \textbf{A. Kolchinsky}, J.A. Owen, ``A space–time tradeoff for implementing a function with master equation dynamics'', \emph{Nature Communications}, 2019. 
\href{https://rdcu.be/bwX2T}{link}
\href{https://arxiv.org/abs/1708.08494}{arxiv} 


J.A. Owen, \textbf{A. Kolchinsky}, D.H. Wolpert, ``Number of hidden states needed to physically implement a given conditional distribution'', \emph{New Journal of Physics}, 2019. 
\href{https://doi.org/10.1088/1367-2630/aaf81d}{link}
\href{https://arxiv.org/abs/1709.00765}{arxiv} 

\textbf{A. Kolchinsky}, B.D. Tracey, ``Estimating Mixture Entropy with Pairwise Distances'', \emph{Entropy}, 2017.
\href{http://www.mdpi.com/1099-4300/19/7/361}{link}
\href{https://arxiv.org/abs/1706.02419}{arxiv}
\href{https://github.com/btracey/mixent/}{code}

C. Gokler, \textbf{A. Kolchinsky}, Z. Liu, I. Marvian , P. Shor, O. Shtanko, K. Thompson, D. Wolpert, S. Lloyd, ``When is a bit worth much more than $kT \ln 2$?'', arXiv:1705.09598, 2017. \href{https://arxiv.org/abs/1705.09598}{arxiv}

\textbf{A. Kolchinsky}, I. Marvian, C. Gokler, Z. Liu, P. Shor, O. Shtanko, K. Thompson, D. Wolpert, S. Lloyd, ``Maximizing free energy gain'', arXiv:1705.00041, 2017. \href{https://arxiv.org/abs/1705.00041}{arxiv}


\textbf{A. Kolchinsky} and L.M. Rocha, ``Prediction and modularity in dynamical systems,'' \emph{Proc
of European Conf. on the Synthesis and Simulation of Living Systems (ECAL)}, 2011. 
\href{http://arxiv.org/abs/1106.3703}{arxiv}


\section{\textsc{Grants}}
9/2019 - Foundational Questions Institute (FQXi), 
``The role of constraints in the thermodynamics of intelligence'' (FQXi-RFP-IPW-1912),
\$118,100, Co-Investigator

8/2016 - Foundational Questions Institute (FQXi), 
``Observers as self-maintaining non-equilibrium systems'' (FQXi-RFP-1622),
\$128,319, Co-Investigator


\section{\textsc{Synergistic Activities}}

\begin{itemize}[leftmargin=*]
\item Currently organizing special issue on ``Thermodynamics and Information Theory of Living Systems'' in \emph{Entropy} journal \href{https://www.mdpi.com/journal/entropy/special_issues/thermodynamics_living_systems}{link} 

\item Lecturer at the SFI Complex Systems Summer School, Santa Fe, NM

\item Mentoring several students for the Santa Fe Institute REU program: \\
\emph{Nicolas Freitas} (Summer 2018), Project: ``Scaling of Information in Biochemical Systems''\\
\emph{Francis Cavanna} (Summer 2017), Project: ``Investigating the relationship between criticality and Landauer costs using the Ising model''

\item Organized a three-day workshop on ``Thermodynamics, evolution, and inference through the lens of information theory'' (w/ V. Ferdinand) at Seoul National University, Seoul, South Korea, 11/2017

\end{itemize}

% \section{\textsc{Teaching}}

% \textbf{Invited Lectures}

% 6/2019 - Santa Fe Institute Complex Systems Summer School, Santa Fe, NM

% \vspace{5pt}

% \textbf{Workshops}

% 3/2019 - Santa Fe Institute, Santa Fe, NM\\
% ``Machine learning with TensorFlow''

% 6/2017, 6/2018 - Santa Fe Institute, Santa Fe, NM\\
% Introduction to programming and data analysis in Python (w/ V. Ferdinand)

% 11/2017 - Seoul National University, Seoul \\
% ``Thermodynamics, evolution, and inference through the lens of information theory'' (w/ V. Ferdinand)

% 11/2017 - ACtioN/Trustee Meeting, Santa Fe Institute, Santa Fe, NM\\
% ``Machine learning: A guide for the perplexed'' (w/ B. D. Tracey)


\section{\textsc{Scientific \& management performance on prior research efforts}}

Artemy Kolchinsky's research lies the intersection of nonequilibrium statistical physics,  
information theory, and theoretical biology. He is particularly interested in understanding the physics of
 autonomous systems that use information to maintain themselves out of equilibrium. He has published numerous papers in these fields, and is
uniquely situated to advance an integrative approach that applies nonequilibrium statistical physics to origin of life research. 
From a management perspective, Artemy Kolchinsky is an early-career researcher, and this would be his first 
time serving as a grant PI. Nonetheless, Kolchinsky participated in the preparation and execution
of several grants while at the Santa Fe Institute (with Prof. David H. Wolpert serving as PI). 
This included the two FQXi grants mentioned above, 
as well as a large scale NSF 
grant (CHE-1648973; ``INSPIRE: Tradeoffs in the Thermodynamics of Computation:
A New Paradigm for Biological Information-Processing'', which funded Kolchinsky as a 
postdoctoral researcher).  Kolchinsky was centrally involved in
the proposal writing, scientific contributions (resulting in numerous acknowledged publications), and reporting for these grants. For the multi-institution 
NSF grant mentioned above, Kolchinsky was also involved in coordinating between researchers at different institutions.


% \section{\textsc{Advising}}
% \emph{Nicolas Freitas}, Santa Fe Institute REU Program, Santa Fe, NM, June-August, 2018 \\
% Project: ``Scaling of Information in Biochemical Systems''

% \emph{Francis Cavanna}, Santa Fe Institute REU Program, Santa Fe, NM, June-August, 2017 \\
% %\hangindent=10pt Mentored a junior from University of Dallas during a summer project through Santa Fe Institute's Research Experiences for Undergraduates program.\\
% Project: ``Investigating the relationship between criticality and Landauer costs using the Ising model''


\end{resume}
\end{document}
