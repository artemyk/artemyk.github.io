\documentclass[margin,line,centered]{res}
\usepackage[utf8]{inputenc}
\usepackage[unicode=true]{hyperref}
\setlength{\parskip}{.35em}
\newsectionwidth{7.5em}
\usepackage{mathptmx}
\usepackage[scaled=.90]{helvet}
\usepackage{courier}

\makeatletter
%%%%%%%%%%%%%%%%%%%%%%%%%%%%%% User specified LaTeX commands.
\usepackage{xcolor}
\definecolor{darkblue}{rgb}{0,0,0.5}
\usepackage{hyperref}

\hypersetup{colorlinks=true,linkcolor=darkblue,urlcolor=darkblue}

\oddsidemargin -.45in
\evensidemargin -.45in

\textwidth=6.18in

\itemsep=0in
\parsep=0in

\addtolength{\topmargin}{0in}
\addtolength{\textheight}{-.1in}


% if using pdflatex:
\setlength{\pdfpagewidth}{\paperwidth}
\setlength{\pdfpageheight}{\paperheight} 

\newenvironment{list1}{
  \begin{list}{\ding{113}}{%
      \setlength{\itemsep}{0in}
      \setlength{\parsep}{0in} \setlength{\parskip}{0in}
      \setlength{\topsep}{0in} \setlength{\partopsep}{0in} 
      \setlength{\leftmargin}{0.17in}}}{\end{list}}
\newenvironment{list2}{
  \begin{list}{$\bullet$}{%
      \setlength{\itemsep}{0in}
      \setlength{\parsep}{0in} \setlength{\parskip}{0in}
      \setlength{\topsep}{0in} \setlength{\partopsep}{0in} 
      \setlength{\leftmargin}{0.2in}}}{\end{list}}

\sloppy

\makeatother

\begin{document}
\name{Artemy Kolchinsky \vspace*{.1in}}
\address{Universal Biology Institute, University of Tokyo}

\begin{resume}

\section{\textsc{Contact}}

\begin{ncolumn}{2}
E-mail: \href{mailto:artemyk@gmail.com}{artemyk@gmail.com} & Google Scholar: \href{https://scholar.google.com/citations?user=RmRwJJIAAAAJ\&sortby=pubdate}{link}\\
\hspace{7pt}  Web: \href{https://artemyk.github.io}{https://artemyk.github.io} & \hspace{30pt} GitHub: \href{https://github.com/artemyk/}{@artemyk}
\end{ncolumn}

\section{\textsc{Education}}

\textbf{Indiana University} (Bloomington, IN, USA), 2015\\
Ph.D. in Informatics (focus in Complex Systems), Minor in Cognitive
Science\\
Thesis: ``Measuring Scales: Integration and Modularity in Complex
Systems''\\
Committee: Luis\,M.\,Rocha\,(chair), Yong-Yeol\,Ahn, Randall\,Beer, Alessandro\,Flammini, Olaf\,Sporns

\textbf{New York University} (New York, NY, USA), 2004 \\
B.A. Magna Cum Laude, Individualized Study (concentration in Complex
Systems)

%\vspace{10pt}

\section{\textsc{Academic Positions}}

\textbf{Universitat Pompeu Fabra} (Barcelona, Spain), June 2023- \\
Marie Curie postdoctoral fellow


\textbf{University of Tokyo} (Tokyo, Japan), Jan 2022-May 2023 \\
Project researcher at the Universal Biology Institute

\textbf{Santa Fe Institute} (Santa Fe, NM, USA), Dec 2015-Dec 2021 \\
Postdoctoral fellow with postdoctoral advisor David H. Wolpert

\textbf{Instituto Gulbenkian de Ciência} (Oeiras, Portugal), 2009-2010 and Summer 2008/2011/2012 \\
Visiting researcher at FLAD Computational Biology Collaboratorium

\textbf{Indiana University} (Bloomington, IN, USA), Sep 2011-May 2015\\
Research assistant with Ph.D. advisor Luis M. Rocha 
%\item \textbf{Research assistant}, NIH-funded project on text mining biomedical literature in PubMed. Indiana University, Bloomington, IN. Supervisor: Luis M. Rocha. 2014-2015
%\item \textbf{Research assistant}, Project for exploring collaborative filtering methods for large-scale job recommendation. Indiana University, Bloomington, IN. Supervisor: Luis M. Rocha. 2012-2013
%\item \textbf{Research assistant}, Indiana University-funded project on discovering drug-drug interactions by text mining biomedical literature. Indiana University, Bloomington, IN. Supervisor: Luis M. Rocha. 2011-2012
%\item \textbf{Freelance programmer}, Developed a variety of database-backed
%web applications. 1998-2007.

\section{\textsc{Industry}}

\textbf{LinkedIn Corporation} (Mountain View, CA, USA), Summer 2014\\
Data science internship. Supervisor: Mathieu Bastian

%\vspace{10pt}
\section{\textsc{Publications}}
%!TEX root = cv.tex

\newcommand{\pdflink}[1]{\href{https://artemyk.github.io/assets/pdf/papers/#1.pdf}{pdf}}

\textbf{A. Kolchinsky}, ``Generalized Zurek's bound on the cost of an individual classical or quantum computation'', \emph{Physical Review E}, 2023. \pdflink{Kolchinsky_2023_GeneralizedZureksBound}

N. Ohga, S. Ito, \textbf{A. Kolchinsky}, ``Thermodynamic bound on the asymmetry of cross-correlations'', \emph{Physical Review Letters}, 2023. (\emph{Editors' Suggestion}; \emph{Featured in \href{https://physics.aps.org/articles/v16/142}{Physics}}) \pdflink{Ohga_Ito_Kolchinsky_2023_ThermodynamicBoundOnTheAsymmetryOfCrossCorrelations}

K. Yoshimura, \textbf{A. Kolchinsky}, A. Dechant, S. Ito, 
``Housekeeping and excess entropy production for general nonlinear dynamics'', 
\emph{Physical Review Research}, 2023. \pdflink{Yoshimura_PRR_2023}

F.C. Sheldon, \textbf{A. Kolchinsky}, F. Caravelli, ``Computational capacity of LRC, memristive, and hybrid reservoirs'', \emph{Physical Review E}, 2022. \pdflink{Sheldon_LRC_PRE_2022}

\textbf{A. Kolchinsky}, ``A Novel Approach to the Partial Information Decomposition'', \emph{Entropy}, 2022.
\pdflink{Kolchinsky_2022_Novel_Approach_to_the_PID} \href{https://github.com/artemyk/redundancy}{code} 

\textbf{A. Kolchinsky} and D.H. Wolpert, ``Dependence of integrated, instantaneous, and fluctuating entropy production on the initial state in quantum and classical processes'', \emph{Physical Review E}, 2021. \pdflink{Kolchinsky_Wolpert_2021_Mismatch_cost}

\textbf{A. Kolchinsky} and D.H. Wolpert, ``Work, entropy production, and thermodynamics of information under protocol constraints'', \emph{Physical Review X}, 2021. \pdflink{Kolchinsky_Wolpert_2021_PRX_Constraints}

\textbf{A. Kolchinsky} and D.H. Wolpert, ``Entropy production given constraints on the energy functions'', \emph{Physical Review E}, 2021. \pdflink{Kolchinsky and Wolpert - 2021 - Entropy production given constraints on the energy functions}

\textbf{A. Kolchinsky} and D.H. Wolpert, ``Thermodynamic costs of Turing Machines'', \emph{Physical Review Research}, 2020. % 2(3), 033312,  August 2020
\pdflink{Kolchinsky_Wolpert_2020_Thermodynamic costs of Turing machines}

D.H. Wolpert and \textbf{A. Kolchinsky}, ``The thermodynamics of computing with circuits'', \emph{New Journal of Physics}, 2020.  % 22, 063047, June 2020
\pdflink{Wolpert and Kolchinsky - 2020 - Thermodynamics of computing with circuits}

\textbf{A. Kolchinsky} and B. Corominas-Murtra, ``Decomposing information into copying versus transformation'', 
\emph{Royal Society Interface}, 2020. % 17(162), 20190623,  January 2020
\pdflink{Kolchinsky_Corominas-Murtra_2020_Decomposing information into copying versus transformation}

A.M. Saxe, Y. Bansal, J. Dapello, M. Advani, \textbf{A. Kolchinsky}, B.D. Tracey, D.D. Cox, 
``On the information bottleneck theory of deep learning'',  \emph{Journal of Statistical Mechanics}, 2019. 
\pdflink{Saxe et al_2019_On the information bottleneck theory of deep learning} 
\href{https://github.com/artemyk/ibsgd/tree/iclr2018}{code} % 124020, Dec 2019

\textbf{A. Kolchinsky}, B.D. Tracey, D.H. Wolpert, ``Nonlinear information bottleneck'', \emph{Entropy}, 2019.
\pdflink{Kolchinsky et al_2019_Nonlinear information bottleneck} (\href{https://www.mdpi.com/journal/entropy/awards/1209}{{\emph{Entropy} 2021 Best Paper Award}})

A. Berdahl, C. Brelsford, C. De Bacco, M. Dumas, V. Ferdinand, J.A. Grochow, L. Hébert-Dufresne,
Y. Kallus, C.P. Kempes, \textbf{A. Kolchinsky}, D. B. Larremore,
E. Libby, E.A. Power, C.A. Stern, B.D.Tracey, ``Dynamics of beneficial epidemics'', \emph{Scientific Reports}, 2019. 
\pdflink{Berdahl et al_2019_Dynamics of beneficial epidemics}

E.A. Hobson, V. Ferdinand, \textbf{A. Kolchinsky}, J. Garland, 
``Rethinking animal social complexity measures with the help of complex systems concepts'', 
\emph{Animal Behaviour}, 2019. 
\pdflink{Hobson et al_2019_Rethinking animal social complexity measures with the help of complex systems}

\textbf{A. Kolchinsky}, B.D. Tracey, S. Van Kuyk, ``Caveats for information bottleneck in deterministic scenarios'', \emph{International Conf on Learning Representations (ICLR)}, 2019. 
\pdflink{Kolchinsky et al_2019_Caveats for information bottleneck in deterministic scenarios} 
\href{https://github.com/artemyk/ibcurve}{code}

D.H. Wolpert, \textbf{A. Kolchinsky}, J.A. Owen, ``A space–time tradeoff for implementing a function with master equation dynamics'', \emph{Nature Communications}, 2019. 
\pdflink{Wolpert et al_2019_A space–time tradeoff for implementing a function with master equation dynamics}

A. Avena-Koenigsberger, X. Yan, \textbf{A. Kolchinsky}, M. van den Heuvel, P. Hagmann, O. Sporns, 
``A spectrum of routing strategies for brain networks'', \emph{PLoS Computational Biology}, 2019. 
\pdflink{Avena-Koenigsberger et al_2019_A spectrum of routing strategies for brain networks}

J.A. Owen, \textbf{A. Kolchinsky}, D.H. Wolpert, ``Number of hidden states needed to physically implement a given conditional distribution'', \emph{New Journal of Physics}, 2019. (\href{https://iopscience.iop.org/article/10.1088/1367-2630/ab60f8}{correction}) 
\pdflink{Owen et al_2018_Number of hidden states needed to physically implement a given conditional} 

\textbf{A. Kolchinsky} and D.H. Wolpert, 
``Semantic information, autonomous agency, and nonequilibrium statistical physics'', 
\emph{Royal Society Interface Focus}, 2018. 
\pdflink{Kolchinsky_Wolpert_2018_Semantic information, autonomous agency and non-equilibrium statistical physics} 
\href{https://github.com/artemyk/semantic_information/}{code}

A.M. Saxe, Y. Bansal, J. Dapello, M. Advani, \textbf{A. Kolchinsky}, B.D. Tracey, D.D. Cox, 
``On the information bottleneck theory of deep learning'', \emph{International Conf on Learning Representations (ICLR)}, 2018. 
\pdflink{Saxe et al_2018_ICLR} \href{https://github.com/artemyk/ibsgd/tree/iclr2018}{code}

\textbf{A. Kolchinsky}, N. Dhande, K. Park, Y.Y. Ahn, ``The Minor Fall, the Major Lift: Inferring emotional valence of musical chords through lyrics'', \emph{Royal Society Open Science}, 2017. 
\pdflink{Kolchinsky et al_2017_The Minor fall, the Major lift} 
\href{https://doi.org/10.6084/m9.figshare.5413060.v1.}{data} 
\href{https://github.com/artemyk/chordsentiment}{code}

\textbf{A. Kolchinsky}, D.H. Wolpert, ``Dependence of dissipation on the initial distribution over states'',
\emph{Journal of Statistical Mechanics}, 2017. 
\pdflink{Kolchinsky_Wolpert_2017_Dependence of dissipation on the initial distribution over states}

\textbf{A. Kolchinsky}, B.D. Tracey, ``Estimating mixture entropy with pairwise distances'', \emph{Entropy}, 2017.
(\href{https://www.mdpi.com/1099-4300/19/11/588}{correction}) 
\pdflink{Kolchinsky_Tracey_2017_Estimating Mixture Entropy with Pairwise Distances} 
\href{https://github.com/btracey/mixent/}{code}


\textbf{A. Kolchinsky}, A.J. Gates, L.M. Rocha, ``Modularity and
the spread of perturbations in complex dynamical systems,'' \emph{Physical Review E}, 2015. 
\pdflink{Kolchinsky et al_2015_Modularity and the spread of perturbations in complex dynamical systems}
\href{https://github.com/artemyk/perturbationmodularity/}{code}

\textbf{A. Kolchinsky}, A. Lourenço, H. Wu, L. Li, L.M. Rocha,
``Extraction of pharmacokinetic evidence of drug-drug interactions
from the literature,'' \emph{PLOS One}, 2015. 
\pdflink{Kolchinsky et al_2015_Extraction of Pharmacokinetic Evidence of Drug–Drug Interactions from the}

\textbf{A. Kolchinsky}, M.P. van den Heuvel, A. Griffa, P. Hagmann, L.M. Rocha, O. Sporns, J. Goñi, ``Multi-scale
integration and predictability in resting state brain activity,'' 
\emph{Frontiers in Neuroinformatics}, 2014. 
\pdflink{Kolchinsky et al_2014_Multi-scale integration and predictability in resting state brain activity}

A. Rossi, F.J. Parada, \textbf{A. Kolchinsky}, A. Puce, ``Neural correlates of apparent motion perception of impoverished facial stimuli I: A comparison of ERP and ERSP activity,'' \emph{NeuroImage}, 2014. 
\pdflink{Rossi et al_2014_Neural correlates of apparent motion perception of impoverished facial stimuli}

\textbf{A. Kolchinsky}, A. Lourenço, L. Li, L.M. Rocha, ``Evaluation of linear classifiers on articles containing pharmacokinetic evidence of drug-drug interactions,'' \emph{Proc Pacific Symposium on Biocomputing}, 2013. 
\pdflink{Kolchinsky et al_2013_Evaluation of linear classifiers on articles containing pharmacokinetic}

\textbf{A. Kolchinsky} and L.M. Rocha, ``Prediction and modularity in dynamical systems,'' \emph{Proc
of European Conf. on the Synthesis and Simulation of Living Systems (ECAL)}, 2011. 
\pdflink{Kolchinsky_Rocha_2011_Prediction and modularity in dynamical systems}

\textbf{A. Kolchinsky}, A. Abi-Haidar, J. Kaur, A.A. Hamed, L.M. Rocha, ``Classification of protein-protein interaction full-text documents using text and citation network features,'' \emph{IEEE/ACM Transactions on Computational Biology and Bioinformatics}, 2010. 
\pdflink{Kolchinsky et al_2010_Classification of Protein-Protein Interaction Full-Text Documents Using Text}




\section{\textsc{Popular science}}
%!TEX root = cv.tex

\textbf{A. Kolchinsky}, ``The cost of sending a bit across a living cell'', \emph{Physics Magazine}, 2023. 
\href{https://physics.aps.org/articles/v16/133}{link}



\section{\textsc{Preprints}}
%!TEX root = cv.tex

\textbf{A. Kolchinsky} and D.H. Wolpert, ``The state dependence of integrated, instantaneous, and fluctuating entropy production in quantum and classical processes'', arXiv:2103.05734, 2021. \href{http://arxiv.org/abs/2103.05734}{arxiv}


F.C. Sheldon, \textbf{A. Kolchinsky}, F. Caravelli, ``The computational capacity of memristor reservoirs'', arXiv:2009.00112, 2020. \href{http://arxiv.org/abs/2009.00112}{arxiv}


\textbf{A. Kolchinsky}, ``A novel approach to multivariate redundancy and synergy'', arXiv:1908.08642, 2019. \href{https://arxiv.org/abs/1908.08642}{arxiv}

C. Gokler, \textbf{A. Kolchinsky}, Z. Liu, I. Marvian , P. Shor, O. Shtanko, K. Thompson, D. Wolpert, S. Lloyd, ``When is a bit worth much more than $kT \ln 2$?'', arXiv:1705.09598, 2017. \href{https://arxiv.org/abs/1705.09598}{arxiv}

\textbf{A. Kolchinsky}, I. Marvian, C. Gokler, Z. Liu, P. Shor, O. Shtanko, K. Thompson, D. Wolpert, S. Lloyd, ``Maximizing free energy gain'', arXiv:1705.00041, 2017. \href{https://arxiv.org/abs/1705.00041}{arxiv}



\section{\textsc{Invited\\Talks}}

9/2024 - \emph{Information in Matter Colloquium}, AMOLF, Amsterdam, Netherlands\\
``Generalized free energy for genuine nonequilibrium systems''

7/2024 - \emph{Seminar at Physics Department, University of Barcelona}, Barcelona, Spain\\
``Thermodynamic Bound on the Asymmetry of Cross-Correlations''

2/2024 - \emph{Seminar}, Barcelona Collaboratorium for Modelling and Predictive Biology, Barcelona, Spain\\
``Stochastic thermodynamics: promises and challenges for studying living systems''

11/2023 - \emph{Mini-workshop: Non-equilibrium thermodynamics and biology}, Biofiska Institute, Bilbao, Spain\\
``Stochastic thermodynamics: promises and limitations for studying living matter''

11/2023 - \emph{Seminar}, Basque Center for Applied Mathematics, Bilbao, Spain\\
``Nonequilibrium thermodynamics, cross-correlations, and the isoperimetric inequality''

7/2023 - \emph{Information Engines at the Frontiers of Nanoscale Thermodynamics}, Telluride, CO, USA\\
``Information geometry for nonequilibrium processes''

6/2023 - \emph{Mathematics, Physics \& Machine Learning Webinar}, Instituto Superior Técnico, Lisbon, Portugal\\
``Information geometry for nonequilibrium processes'' (virtual)

6/2023 - \emph{Conference on Decomposing Multivariate Information in Complex Systems}, Max Planck Institute for the Physics of Complex Systems, Dresden, Germany\\
``A Novel Approach to the Partial Information Decomposition''

9/2022 - \emph{Seminar at Dutch Institute of Emergent Phenomena}, University of Amsterdam, Netherlands\\
``Information geometry of fluxes and forces in nonequilibrium thermodynamics'' (virtual)

8/2022 - \emph{Seminar}, Earth-Life Science Institute (ELSI), Tokyo, Japan\\
``Thermodynamics of Darwinian evolution in molecular replicators''

6/2022 - \emph{Evolution of complexity from the statistical physics perspective}, Yerevan Physics Institute, Armenia\\
``Thermodynamics of Darwinian evolution'' (virtual)

6/2022 - \emph{Yagami Statistical Physics Seminar}, Keio University, Japan\\
``Thermodynamics under protocol constraints'' (virtual)

5/2022 - \emph{Workshop on Stochastic Thermodynamics III}, University of Tokyo, Japan\\
``The algorithmic cost of a single classical or quantum computation'' (virtual) % Invited

4/2022 - \emph{Cross Labs Workshop: Is AI Extending the Mind?}, Cross Labs, Japan\\
``Autonomous agents and semantic information'' (virtual) % Invited

2/2022 - \emph{Physics Department Colloquium Series}, University of Rochester, NY, USA\\
``A thermodynamic threshold for Darwinian evolution'' (virtual) % Invited

12/2021 - \emph{Universal Biology Institute Seminar Series}, University of Tokyo, Japan\\
``A thermodynamic threshold for Darwinian evolution'' (virtual) % Invited

02/2021 - \emph{Origins of Life: The Possible and the Actual} workshop, Santa Fe Institute, NM, USA\\
``Fundamental thermodynamic constraints and trade-offs in origin of life'' %Invited

7/2020 - \emph{ICTP Seminar Series}, Abdus Salam International Center for Theoretical Physics, Trieste, Italy\\
``Bounds on entropy production and thermodynamics of information under protocol constraints'' (virtual) % Invited

2/2020 - \emph{AI Seminar Series}, Information Sciences Institute, Los Angeles, CA, USA\\
``Machine Learning through the information bottleneck'' % Invited

7/2019 - \emph{ISTI Seminar Series}, Los Alamos National Lab, Los Alamos, NM, USA\\
``Machine Learning through the information bottleneck''  % Invited

6/2018 - \emph{Connectomics Lecture Series}, Universidad Diego Portales, Santiago, Chile\\
``Machine learning, deep neural networks, and the brain''  % Invited

4/2018 - \emph{Meeting of the Society for the Neural Control of Movement}, Santa Fe, NM, USA\\
``Machine learning, deep neural networks, and the brain''

4/2018 - \emph{SITE Santa Fe} (contemporary art museum), Santa Fe, NM, USA\\
``Life, entropy, and the $2^{{nd}}$ law of thermodynamics''

11/2017 - Seoul National University, South Korea\\
``Science at the Santa Fe Institute'' (w/ V. Ferdinand) % Invited

8/2017 - \emph{Thermodynamics \& Computation: Towards a New Synthesis}, Santa Fe Institute, NM, USA\\
``Statistical physics of Turing Machines'' % Invited

10/2016 - \emph{Statistical Physics, Information Processing and Biology}, Santa Fe Institute, NM, USA \\
``Dependence of dissipation on the initial distribution'' % Invited.

2/2016 - Information Sciences Institute, Los Angeles, CA, USA\\
``Multi-scale integration \& modularity in complex systems'' % Invited
% Video : http://webcastermshd.isi.edu/Mediasite/Play/f413ecae075e40eaa3f6b51123178b791d


% \vspace{5pt}

% \textbf{Contributed}

% 3/2021 - \emph{American Physical Society March Meeting} (virtual)\\
% ``Thermodynamics under protocol constraints'' (w/ D.H. Wolpert)% 

% 6/2020 - \emph{Stochastic thermodynamics in complex systems}, Complexity Science Hub, Vienna, Austria\\
% ``Entropy production \& thermodynamics of information under protocol constraints''

% 5/2019 - \emph{Seminar}, Max Planck Institute for Mathematics in the Sciences, Leipzig, Germany\\
% ``A novel measure of multivariate redundant information''

% 3/2019 - \emph{American Physical Society March Meeting}, Boston, MA\\
% ``Thermodynamics of Turing Machines'' (w/ D.H. Wolpert)% 

% 3/2018 - \emph{American Physical Society March Meeting}, Los Angeles, CA\\
% ``Thermodynamic costs, initial distributions, and Bregman divergences'' (w/ D.H. Wolpert)% 

% 1/2018 - \emph{Information theory and non-equilibrium thermodynamics beyond the Shannon-Gibbs framework}, Complexity Science Hub, Vienna, Austria\\
% ``Entropy in stochastic thermodynamics''

% 12/2017 - \emph{Complexity, Criticality \& Computation International Biannual Symposium}, University of Sydney\\
% ``Grounding semantic information in the dynamics of non-equilibrium systems'' (w/ D.H. Wolpert)

% 8/2017 - \emph{Information Engines at the Frontiers of Nanoscale Thermodynamics}, Telluride, CO\\
% ``Semantic information, observation and non-equilibrium systems'' (w/ D.H. Wolpert) % Invited

% 3/2017 - \emph{American Physical Society March Meeting}, New Orleans, LA\\
% ``Dependence of dissipation on the initial distribution'' (w/ D.H. Wolpert)% 

% 10/2015 - \emph{Information Theory, Ecosystems, \& Schrodinger's Paradox} workshop, Santa Fe Institute\\
% ``Complexity measures for spatially embedded systems''

% 9/2015 - \emph{Conference on Complex Systems 2015}, Tempe, AZ\\
% ``Modularity and the spread of perturbations in complex dynamical systems'' (w/ A.J. Gates, L.M. Rocha)\\
% (awarded ``Honorable Mention Paper by a Contributing Student'')

% 10/2013 - \emph{Indiana Neuroimaging Symposium}, Indiana University, Bloomington, IN\\
% ``Information, space \& structure in the human brain resting state'' [poster] (w/ M.P. van den Heuvel, A. Griffa, P. Hagmann, L.M. Rocha, O. Sporns, J. Goñi)

% 9/2013 - \emph{Guided Self-Organization 6 workshop, European Conf on Complex Systems}, Barcelona, Spain\\
% ``Modularity and dynamical timescales in Boolean Networks'' %(w/ Luis Rocha)

% 3/2013 - \emph{MBI Rhythms and Oscillations Workshop}, Columbus, OH\\
% ``Studying differences in oscillatory synchronization with tensor-factorization'' [poster]
% (w/ F.J. Parada, L.M. Rocha, T. Busey)

% 1/2013 - \emph{Pacific Symposium on Biocomputing}, Big Island, Hawaii\\
% ``Evaluation of linear classifiers on articles containing pharmacokinetic evidence
%  of drug-drug interactions'' %(w/Lourenço, A., Li, L., Rocha, L.M.)

% % Goñi, J., \textbf{Kolchinsky}, A., van den Heuvel, M.P., Griffa, A.,
% % L.M. Rocha, Sporns, O., ``Information, space and
% % structure in the human brain resting state,'' \emph{
% % 12th Granada Seminar on Computational and Statistical Physics} (La
% % Herradura, Spain), 9/2012. {[}presented by Goñi, J.{]} 

% 12/2011 - \emph{Network Frontier Workshop}, Northwestern University, Evanston, IL\\
% ``Prediction and modularity in dynamical systems''

% % Parada, F.J,\textbf{ A. Kolchinsky}, Rossi, A., Busey, T., Sporns,
% % O., Puce, A., ``Spatial organization of EEG cross-frequency
% % coupling in a perceptual task,'' Symposium on ``Human
% % visual perception'',\emph{ Society for Neuroscience} (Washington,
% % DC), 11/2011. {[}presented by Parada, F.J.{]}

% 4/2011 - %``Perception, Sensation, Information, and Ecology - Taking the animal's point of view,'' 
% \emph{CISAB Animal Behavior Conference}, Indiana University, Bloomington, IN\\
% ``The Umwelt, artificial life, and evolution''

% 9/2010 - \emph{Guided Self-Organization 3 work}, Indiana University, Bloomington, IN\\
% ``Identifying dynamical modules in Boolean network models''

% 3/2008 - \emph{Interdisciplinary Symposium on the Mind}, University of Toronto\\
% ``The Expanded Mind: Mental expansion and the intentional stance''


% % Parada, F.J., A. Kolchinsky, Rossi, Puce, A. ``EEG phase-coupling
% % dynamics in apparent motion perception,'' \emph{Organization for
% % Human Brain Mapping 2013} (Seattle, WA), 2013.

% %Parada, F.J., A. Kolchinsky, Rossi, A., Busey, T., Sporns,
% % O., Puce, A. ``From visual awareness to perceptual decision-making:
% % neural correlates of top-down and bottom-up dynamics,'' \emph{MBI
% % Rhythms and Oscillations Workshop} (Columbus, OH), 2013.

% %Rossi, A., Parada, F.J., A. Kolchinsky, Puce, A. ``Neural
% %responses to changes in social attention depicted by biological motion
% %stimuli,'' \emph{Organization for Human Brain Mapping 2012} (Beijing,
% %China), 2012.


\section{\textsc{Awards \& Fellowships}}
%\item Honorable Mention Paper by a Contributing Student, Conference on Complex Systems 2015, Tempe, AZ
% 2010-2015 - Affiliate of IGERT training program in ``Dynamics of brain-body-environment interaction in behavior and cognition'' 
2022 - Marie Curie Individual Fellowship (HORIZON-MSCA-2021-PF-01)\\
``NETOLIFE: Nonequilibrium thermodynamics of the origin of life''

2012 - 2013 - Lilly Graduate Fellowship, Biocomplexity Institute, Indiana University, Bloomington, IN

2007 - 2009 - Eli Lilly Fellowship, Indiana University, Bloomington, IN, 

2004 - Dean's List Gallatin School, New York University, NY

\section{\textsc{Grants}}

2023 - John Templeton Foundation (62828), \$947,926, Co-PI\\
``Goal-directed behavior and the origin of life''

2022 - John Templeton Foundation (62417), \$627,227, Co-PI\\
``Information architectures that enable life: the emergence of meaning''

2019 - Foundational Questions Institute (FQXi-RFP-IPW-1912),  \$118,100, Co-Investigator (PI: David Wolpert)
``The role of constraints in the thermodynamics of intelligence'' 

2016 - Foundational Questions Institute (FQXi-RFP-1622), \$128,319, Co-Investigator (PI: David Wolpert)\\
``Observers as self-maintaining non-equilibrium systems''

2016 - NSF INSPIRE (CHE-1648973), \$999,947, co-author (PI: David Wolpert)\\
``Tradeoffs in the Thermodynamics of Computation: A New Paradigm for Biological Information-Processing''

2016 - NSF (1620462), \$770,000, co-author (PI: David Wolpert)\\
``Information Networks and the Evolution of Social Organizations'' 



\section{\textsc{Teaching}}

\textbf{Online courses}\\
2018 - ``Fundamentals of machine learning'' (w/ B.D. Tracey), Complexity Explorer from the Santa Fe Institute\\
Designed and delivered a 12-part online course on basics of machine learning {[}\href{https://www.complexityexplorer.org/courses/81-fundamentals-of-machine-learning}{link}{]}

\vspace{5pt}


\textbf{Workshops, lectures, and tutorials}\\
6/2019 - Santa Fe Institute Complex Systems Summer School, NM, USA\\
Two lecture series on machine learning and its research frontiers

3/2019 - Santa Fe Institute, NM, USA\\
Tutorial on ``Machine learning with TensorFlow''

6/2017, 6/2018 - Santa Fe Institute, NM, USA\\
Tutorial on introduction to programming and data analysis in Python (w/ V. Ferdinand)

11/2017 - Seoul National University, Seoul, South Korea \\
Week-long workshop on ``Thermodynamics, evolution, and inference through the lens of information theory'' (w/ V. Ferdinand)

11/2017 - ACtioN/Trustee Meeting, Santa Fe Institute, NM, USA\\
Lecture on ``Machine learning: A guide for the perplexed'' (w/ B. D. Tracey)

5/2010 - Instituto Gulbenkian de Ciência, Oeiras, Portugal\\
Assisted with week-long ``Bayesian brain'' educational module

\vspace{5pt}

\textbf{Teaching Assistant at Indiana University, Bloomington, IN}\\
Spring 2014 - ``I400 Large-scale Social Phenomena'' {[}\href{http://tuvalu.santafe.edu/~simon/page11/page11.html}{link}{]} 

Spring 2011 - ``I201 Math and logic foundations of Informatics''

Fall 2010 - ``I485 Biologically Inspired Computing'' {[}\href{https://web.archive.org/web/20110310071102/http://www.informatics.indiana.edu/rocha/i-bic/index.html}{link}{]}\\
Developed and implemented a 5-part computational lab series

Fall 2008-Spring 2009 - ``I210 Information Infrastructure'' (Python programming)



\section{\textsc{Advising}}
\emph{Nicolas Freitas}, Santa Fe Institute REU Program, Santa Fe, NM, June-August, 2018 \\
Project: ``Scaling of information in biochemical systems''

\emph{Francis Cavanna}, Santa Fe Institute REU Program, Santa Fe, NM, June-August, 2017 \\
%\hangindent=10pt Mentored a junior from University of Dallas during a summer project through Santa Fe Institute's Research Experiences for Undergraduates program.\\
Project: ``Investigating the relationship between criticality and Landauer costs using the Ising model''

\section{\textsc{Academic\\service}}

\textbf{Guest editor}: \emph{Entropy} special issue on ``Thermodynamics and Information Theory of Living Systems'' \href{https://www.mdpi.com/journal/entropy/special_issues/thermodynamics_living_systems}{link}

\textbf{Reviewer}\\
Physics: \emph{New J of Physics}, \emph{PRL}, \emph{PRR}, \emph{PRE}, \emph{Frontiers in Physics}, \emph{J of Physics A}, \emph{Physica A}\\
Biology: \emph{J R Soc Proc B}, \emph{PLoS Comp Bio}, \emph{Theory in the Biosciences}\\
Information theory and machine learning: \emph{Entropy}, \emph{ICLR}, \emph{IEEE Trans on Pattern Analysis and Machine Intelligence}, \emph{Kybernetika}, \emph{Applied Sciences}

\textbf{Other}\\
2008-2013 - Started and ran a weekly discussion group on complexity, dynamical systems, and embodiment in cognitive science, Indiana University, Bloomington, IN \href{http://apophenia.wikidot.com/}{link}
%Co-organized discussion group about Friston's ``Free energy principle'', Instituto Gulbenkian de Ciencia, Oeiras, Portugal, Spring 2010\\
%Volunteer organizer for Society for Psychology and Philosophy Conference, Indiana University, Bloomington, IN, 6/2009


\section{\textsc{Skills}}


%\emph{Technical Skills}\\
%Machine learning, multivariate statistical analysis, network analysis,
%information theory

\hangindent=10pt \emph{Programming}: Python, MATLAB, C, C++, R, Java\\
 machine learning with TensorFlow, web programming, databases/SQL, scalable computing % (Hadoop, PIG, Scala)

\emph{Languages}: Fluency: English, Russian, Spanish / Basic: Portuguese

\end{resume}
\end{document}
